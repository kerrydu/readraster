. 
. ********Example of Using readraster Package************
. cap findfile readraster.sthlp
{\smallskip}
.     if _rc {\lbr}
.     display "change the working directory to the folder where readraster packa
> ge is located"
.     exit
.     {\rbr}
{\smallskip}
. 
. cap findfile DMSP-like2020.tif
{\smallskip}
. if _rc {\lbr}
.    display "change the working directory to the folder where readraster packag
> e is located"
.    exit
. {\rbr}
{\smallskip}
. ************Set up for Readraster************
. 
. cap findfile gt-main-32.0.jar
{\smallskip}
. if _rc {\lbr}
.    display "using setup.do to install Java dependencies"
.    display "downloading the Java dependencies requires dozen minutes and might
>  fail due to network issues"
.    display `"if it fails, please try again, or download the Java dependencies 
> manually as instructed in georools_init (see {\lbr}view "geotools_init.sthlp":help 
> geotools_init{\rbr})"'
.    do setup.do
. {\rbr}
{\smallskip}
. 
. cap which geoplot
{\smallskip}
. if _rc {\lbr}
.    ssc install geoplot, replace
. {\rbr}
{\smallskip}
. cap which lmoremata.mlib
{\smallskip}
. if _rc {\lbr}
.     ssc install moremata, replace
. {\rbr}
{\smallskip}
. cap which heatplot
{\smallskip}
. if _rc {\lbr}
.     ssc install heatplot, replace
. {\rbr}
{\smallskip}
. cap which palettes.hlp
{\smallskip}
. if _rc {\lbr}
.     ssc install palettes, replace
. {\rbr}
{\smallskip}
. 
. 
. 
. set scheme sj, permanently
({\bftt{set scheme}} preference recorded)
{\smallskip}
. 
. **********************************************************
. * downloading nc file
. cap findfile tas_day_BCC-CSM2-MR_ssp245_r1i1p1f1_gn_2050.nc
{\smallskip}
. 
. if _rc{\lbr}
.         local url = "https://nex-gddp-cmip6.s3-us-west-2.amazonaws.com/" + ///
>                                 "NEX-GDDP-CMIP6/BCC-CSM2-MR/ssp245/r1i1p1f1/ta
> s/" + ///
>                                 "tas_day_BCC-CSM2-MR_ssp245_r1i1p1f1_gn_2050.n
> c"
.                                 
.         copy `"`url'"' .
. {\rbr}
{\smallskip}
. **************************Example of GeoTIFF********************************
. ************5.1 Display the Metadata************
. 
. //Display the Metadata of the GeoTIFF File
. gtiffdisp DMSP-like2020.tif
{\smallskip}
=== Band Information ===
Number of bands: 1
Band 1 : GRAY_INDEX           | NoData: Not defined
{\smallskip}
=== Spatial Characteristics ===
X range: [-2643772.8565 {\tytilde} 2212227.1435]
Y range: [1871896.5263 {\tytilde} 5926896.5263]
Resolution: X=1000.0000 units/pixel, Y=1000.0000 units/pixel
{\smallskip}
=== Coordinate System ===
CRS Name: PCS Name = WGS_1984_Albers
CRS WKT: PROJCS["PCS Name = WGS_1984_Albers", 
  GEOGCS["WGS 84", 
    DATUM["World Geodetic System 1984", 
      SPHEROID["WGS 84", 6378137.0, 298.257223563, AUTHORITY["EPSG","7030"]], 
      AUTHORITY["EPSG","6326"]], 
    PRIMEM["Greenwich", 0.0, AUTHORITY["EPSG","8901"]], 
    UNIT["degree", 0.017453292519943295], 
    AXIS["Geodetic longitude", EAST], 
    AXIS["Geodetic latitude", NORTH], 
    AUTHORITY["EPSG","4326"]], 
  PROJECTION["Albers_Conic_Equal_Area"], 
  PARAMETER["central_meridian", 105.0], 
  PARAMETER["latitude_of_origin", 0.0], 
  PARAMETER["standard_parallel_1", 25.0], 
  PARAMETER["false_easting", 0.0], 
  PARAMETER["false_northing", 0.0], 
  PARAMETER["standard_parallel_2", 47.0], 
  UNIT["m", 1.0], 
  AXIS["Easting", EAST], 
  AXIS["Northing", NORTH]]
{\smallskip}
=== Units ===
X unit: m
Y unit: m
{\smallskip}
=== Filtered Metadata ===
image_height                : 4055
image_min_x_coord           : 0
image_width                 : 4856
image_min_y_coord           : 0
{\smallskip}
. 
. ************5.2 Import Raster Data into Stata************
. 
. //Read the GeoTIFF file for a specific region
. spshape2dta hunan.shp, replace 
  (importing .shp file)
  (importing .dbf file)
  (creating _ID spatial-unit id)
  (creating _CX coordinate)
  (creating _CY coordinate)
{\smallskip}
  file hunan_shp.dta created
  file hunan.dta     created
{\smallskip}
. use "hunan.dta",clear
{\smallskip}
. crsconvert _CX _CY, gen(alber) from(hunan.shp) to(DMSP-like2020.tif)
Converting coordinates from CRS:
Source CRS: GEOGCS["GCS_WGS_1984", 
  DATUM["D_WGS_1984", 
    SPHEROID["WGS_1984", 6378137.0, 298.257223563]], 
  PRIMEM["Greenwich", 0.0], 
  UNIT["degree", 0.017453292519943295], 
  AXIS["Longitude", EAST], 
  AXIS["Latitude", NORTH]]
Target CRS: PROJCS["PCS Name = WGS_1984_Albers", 
  GEOGCS["WGS 84", 
    DATUM["World Geodetic System 1984", 
      SPHEROID["WGS 84", 6378137.0, 298.257223563, AUTHORITY["EPSG","7030"]], 
      AUTHORITY["EPSG","6326"]], 
    PRIMEM["Greenwich", 0.0, AUTHORITY["EPSG","8901"]], 
    UNIT["degree", 0.017453292519943295], 
    AXIS["Geodetic longitude", EAST], 
    AXIS["Geodetic latitude", NORTH], 
    AUTHORITY["EPSG","4326"]], 
  PROJECTION["Albers_Conic_Equal_Area"], 
  PARAMETER["central_meridian", 105.0], 
  PARAMETER["latitude_of_origin", 0.0], 
  PARAMETER["standard_parallel_1", 25.0], 
  PARAMETER["false_easting", 0.0], 
  PARAMETER["false_northing", 0.0], 
  PARAMETER["standard_parallel_2", 47.0], 
  UNIT["m", 1.0], 
  AXIS["Easting", EAST], 
  AXIS["Northing", NORTH]]
{\smallskip}
. 
. qui sum alber_CX
{\smallskip}
. local maxX = r(max)+2000
{\smallskip}
. local minX = r(min)-2000
{\smallskip}
. 
. qui sum alber_CY
{\smallskip}
. local maxY = r(max)+2000
{\smallskip}
. local minY = r(min)-2000
{\smallskip}
. 
. gtiffread DMSP-like2020.tif, origin(1 1) size(-1 1) clear 
{\smallskip}
. gen n=_n
{\smallskip}
. sum n if y>`minY' \& y<`maxY'
{\smallskip}
    Variable {\VBAR}        Obs        Mean    Std. dev.       Min        Max
\HLI{13}{\PLUS}\HLI{57}
           n {\VBAR}        400      3006.5    115.6143       2807       3206
{\smallskip}
. local start_row = r(min)
{\smallskip}
. local n_rows = r(N)
{\smallskip}
. 
. gtiffread DMSP-like2020.tif, origin(1 1) size(1 -1) clear 
{\smallskip}
. gen n=_n
{\smallskip}
. sum n if x>`minX' \& x<`maxX'
{\smallskip}
    Variable {\VBAR}        Obs        Mean    Std. dev.       Min        Max
\HLI{13}{\PLUS}\HLI{57}
           n {\VBAR}        385        3292    111.2842       3100       3484
{\smallskip}
. local start_col = r(min)
{\smallskip}
. local n_cols = r(N)
{\smallskip}
. 
. gtiffread DMSP-like2020.tif, origin(`start_row' `start_col') ///
>                              size(`n_rows' `n_cols') clear
{\smallskip}
. 
. save DMSP-like2020.dta,replace
file{\bftt{ DMSP-like2020.dta}} saved
{\smallskip}
. 
. // heatplot
. use DMSP-like2020.dta, clear
{\smallskip}
. heatplot value y x, color(Greys, reverse) level(6) ///
>                 xlabel(, angle(45)) graphr(c(white))
{\smallskip}
. 
. graph save gragh1, replace
file {\bftt{gragh1.gph}} saved
{\smallskip}
. 
. 
. ********5.3 Calculating Average Nighttime Light Intensity for hunan********
. zonalstats DMSP-like2020.tif using hunan.shp, stats("avg") clear
GeoTIFF CRS detected: PCS Name = WGS_1984_Albers. User-provided CRS is ignored.
Shapefile CRS: GCS_WGS_1984
Reprojecting shapefile from GCS_WGS_1984 to PCS Name = WGS_1984_Albers
Shapefile bounds for raster reading: ReferencedEnvelope[373271.4002242747 : 9187
> 28.259838356, 2596264.786161866 : 3202467.2929103803] DefaultProjectedCRS[PCS 
> Name = WGS_1984_Albers] AXIS["Easting", EAST] AXIS["Northing", NORTH]
Raster envelope: ReferencedEnvelope[-2643772.8565207715 : 2212227.1434792285, 18
> 71896.5262559392 : 5926896.526255939] DefaultProjectedCRS[PCS Name = WGS_1984_
> Albers] AXIS["Easting", EAST] AXIS["Northing", NORTH]
Using intersection bounds: ReferencedEnvelope[373271.4002242747 : 918728.2598383
> 56, 2596264.786161866 : 3202467.2929103803] DefaultProjectedCRS[PCS Name = WGS
> _1984_Albers] AXIS["Easting", EAST] AXIS["Northing", NORTH]
Successfully created optimized read parameters
Successfully read raster data with optimization
Total features: 14
Created string variable: z_ShiName (length 16)
Created numeric variable: z_Shape_Leng
Created numeric variable: z_Shape_Area
Created string variable: z_Name (length 16)
Created numeric variable: z_lon
Created numeric variable: z_lat
Created numeric variable: avg
Data successfully exported to Stata dataset.
{\smallskip}
. list z_Name avg
{\smallskip}
     {\TLC}\HLI{25}{\TRC}
     {\VBAR}      z_Name         avg {\VBAR}
     {\LFTT}\HLI{25}{\RGTT}
  1. {\VBAR}     Changde   3.2560185 {\VBAR}
  2. {\VBAR}    Chenzhou   2.0607766 {\VBAR}
  3. {\VBAR}    Hengyang   3.2285136 {\VBAR}
  4. {\VBAR}     Huaihua   1.2616182 {\VBAR}
  5. {\VBAR}       Loudi   3.7707743 {\VBAR}
     {\LFTT}\HLI{25}{\RGTT}
  6. {\VBAR}    Shaoyang   1.7991827 {\VBAR}
  7. {\VBAR}    Xiangtan   7.2288051 {\VBAR}
  8. {\VBAR}     Xiangxi    1.458301 {\VBAR}
  9. {\VBAR}      Yiyang   2.4555944 {\VBAR}
 10. {\VBAR}    Yongzhou   2.0996052 {\VBAR}
     {\LFTT}\HLI{25}{\RGTT}
 11. {\VBAR}     Yueyang   3.8674934 {\VBAR}
 12. {\VBAR} Zhangjiajie   1.7823899 {\VBAR}
 13. {\VBAR}    Changsha   11.616603 {\VBAR}
 14. {\VBAR}     Zhuzhou   3.8699787 {\VBAR}
     {\BLC}\HLI{25}{\BRC}
{\smallskip}
. save "hunan_light.dta", replace
file{\bftt{ hunan_light.dta}} saved
{\smallskip}
. 
. use hunan.dta, clear
{\smallskip}
. 
. rename Name z_Name  
{\smallskip}
. merge 1:1 z_Name using hunan_light.dta,nogen
(variable {\bftt{z_Name}} was {\bftt{str11}}, now {\bftt{str16}} to accommodate using data's values)
{\smallskip}
    Result                      Number of obs
    \HLI{41}
    Not matched                             0
    Matched                                14  
    \HLI{41}
{\smallskip}
. save hunan_light.dta, replace
file{\bftt{ hunan_light.dta}} saved
{\smallskip}
. 
. geoframe create region     ///
>          "hunan_light.dta", id(_ID) centroids(_CX _CY) ///
>           shp(hunan_shp.dta) replace
(creating frame {\bftt{region}} from hunan_light.dta)
(creating frame {\bftt{region_shp}} from hunan_shp.dta)
{\smallskip}
            Frame name: {\bftt{region}} [make current]
            Frame type: {\bftt{attribute}}
          Feature type: <none>
         Number of obs: {\bftt{14}}
               Unit ID: {\bftt{_ID}}
           Coordinates: {\bftt{_CX _CY}}
    Linked shape frame: {\bftt{region_shp}}
{\smallskip}
. 
. geoplot ///
>     (area region avg, color(Greys, reverse) ///
>                 level(6, quantile weight(avg))) ///
>     (line region, lwidth(vthin)), legend(position(sw))
{\smallskip}
.  
. graph save gragh2, replace
file {\bftt{gragh2.gph}} saved
{\smallskip}
. 
. ********5.4 Calculate 80km-radius IDW light for cities********
. 
. use "DMSP-like2020.dta", clear
{\smallskip}
. crsconvert x y, gen(wsg84_) from(DMSP-like2020.tif) to(hunan.shp)
Converting coordinates from CRS:
Source CRS: PROJCS["PCS Name = WGS_1984_Albers", 
  GEOGCS["WGS 84", 
    DATUM["World Geodetic System 1984", 
      SPHEROID["WGS 84", 6378137.0, 298.257223563, AUTHORITY["EPSG","7030"]], 
      AUTHORITY["EPSG","6326"]], 
    PRIMEM["Greenwich", 0.0, AUTHORITY["EPSG","8901"]], 
    UNIT["degree", 0.017453292519943295], 
    AXIS["Geodetic longitude", EAST], 
    AXIS["Geodetic latitude", NORTH], 
    AUTHORITY["EPSG","4326"]], 
  PROJECTION["Albers_Conic_Equal_Area"], 
  PARAMETER["central_meridian", 105.0], 
  PARAMETER["latitude_of_origin", 0.0], 
  PARAMETER["standard_parallel_1", 25.0], 
  PARAMETER["false_easting", 0.0], 
  PARAMETER["false_northing", 0.0], 
  PARAMETER["standard_parallel_2", 47.0], 
  UNIT["m", 1.0], 
  AXIS["Easting", EAST], 
  AXIS["Northing", NORTH]]
Target CRS: GEOGCS["GCS_WGS_1984", 
  DATUM["D_WGS_1984", 
    SPHEROID["WGS_1984", 6378137.0, 298.257223563]], 
  PRIMEM["Greenwich", 0.0], 
  UNIT["degree", 0.017453292519943295], 
  AXIS["Longitude", EAST], 
  AXIS["Latitude", NORTH]]
{\smallskip}
. gen n=_n
{\smallskip}
. save "light_china.dta", replace
file{\bftt{ light_china.dta}} saved
{\smallskip}
. 
. use "hunan_city.dta", clear
{\smallskip}
. matchgeop ORIG_FID lat lon using light_china.dta, ///
>                neighbors(n wsg84_y wsg84_x) within(80) gen(distance)
{\smallskip}
. 
. merge m:1 n using light_china.dta, keep(3)
{\smallskip}
    Result                      Number of obs
    \HLI{41}
    Not matched                             0
    Matched                           206,615  (_merge==3)
    \HLI{41}
{\smallskip}
. drop _merge
{\smallskip}
. 
. drop if value==.
(0 observations deleted)
{\smallskip}
. gen weight  = 1/distance
{\smallskip}
. gen weighted_light = value * weight
{\smallskip}
. bysort city: egen sum_weighted_light = total(weighted_light)
{\smallskip}
. bysort city: egen total_weight = total(weight)
{\smallskip}
. gen idw_light = sum_weighted_light / total_weight
{\smallskip}
. 
. duplicates drop city, force
{\smallskip}
Duplicates in terms of city
{\smallskip}
(206,601 observations deleted)
{\smallskip}
. 
. ///IDW interpolated light distributions in Hunan
> rename city Name
{\smallskip}
. merge m:1 Name using hunan.dta
{\smallskip}
    Result                      Number of obs
    \HLI{41}
    Not matched                             0
    Matched                                14  (_merge==3)
    \HLI{41}
{\smallskip}
. 
. save "hunan_light.dta", replace
file{\bftt{ hunan_light.dta}} saved
{\smallskip}
. 
. geoframe create region ///
>    "hunan_light.dta", id(_ID) centroids(_CX _CY) ///
>     shp(hunan_shp.dta) replace
(creating frame {\bftt{region}} from hunan_light.dta)
(creating frame {\bftt{region_shp}} from hunan_shp.dta)
{\smallskip}
            Frame name: {\bftt{region}} [make current]
            Frame type: {\bftt{attribute}}
          Feature type: <none>
         Number of obs: {\bftt{14}}
               Unit ID: {\bftt{_ID}}
           Coordinates: {\bftt{_CX _CY}}
    Linked shape frame: {\bftt{region_shp}}
{\smallskip}
. 
. geoplot ///
>   (area region idw_light , color(Greys, reverse) ///
>           level(6, quantile weight(idw_light))) ///
>   (line region, lwidth(vthin)), legend(position(sw))
{\smallskip}
. 
. graph save gragh3, replace
file {\bftt{gragh3.gph}} saved
{\smallskip}
. 
. **************************Example of NetCDF********************************
. 
. ************6.1 Display the Metadata************
. 
. //Display the Metadata of the NetCDF File
. ncdisp using tas_day_BCC-CSM2-MR_ssp245_r1i1p1f1_gn_2050.nc
{\smallskip}
=== File Structure ===
{\smallskip}
[Global Attributes]
_NCProperties            : version=1|netcdflibversion=4.4.1.1|hdf5libversi...
activity                 : NEX-GDDP-CMIP6
contact                  : Dr. Rama Nemani: rama.nemani@nasa.gov, Dr. Brid...
Conventions              : CF-1.7
creation_date            : 2021-10-04T14:25:12.081609+00:00
frequency                : day
institution              : NASA Earth Exchange, NASA Ames Research Center,...
variant_label            : r1i1p1f1
product                  : output
realm                    : atmos
source                   : BCSD
scenario                 : ssp245
references               : BCSD method: Thrasher et al., 2012, Hydrol. Ear...
version                  : 1.0
tracking_id              : 6ab2a874-d3fe-484a-a7e8-82df448e364f
title                    : BCC-CSM2-MR, r1i1p1f1, ssp245, global downscale...
resolution_id            : 0.25 degree
history                  : 2021-10-04T14:25:12.081609+00:00: install globa...
disclaimer               : This data is considered provisional and subject...
external_variables       : areacella
cmip6_source_id          : BCC-CSM2-MR
cmip6_institution_id     : BCC
cmip6_license            : CC-BY-SA 4.0
_CoordSysBuilder         : ucar.nc2.internal.dataset.conv.CF1Convention
{\smallskip}
[Dimensions]
Name                 Length   Attribute      
time                 365      Coordinate     
lat                  600      Coordinate     
lon                  1440     Coordinate     
{\smallskip}
[Variables]
Name                      Dimensions                     Type           
tas                       time lat lon                   float          
time                      time                           double         
lat                       lat                            double         
lon                       lon                            double         
{\smallskip}
[Groups]
{\smallskip}
. 
. //Display variable metadata with ncdisp
. ///tas variable
> ncdisp tas using tas_day_BCC-CSM2-MR_ssp245_r1i1p1f1_gn_2050.nc
{\smallskip}
=== Dimensions ===
Dimension       Length   Coordinate     
time            365      [Yes]          
lat             600      [Yes]          
lon             1440     [Yes]          
{\smallskip}
=== Scale/Offset Parameters ===
missing value  : 100000002004087730000.000000 (Type: float)
 FillValue     : 100000002004087730000.000000 (Type: float)
{\smallskip}
=== Attributes ===
missing_value       : 1.0E20
cell_measures       : area: areacella
cell_methods        : area: mean time: maximum (interval: 5 minutes)
original_name       : TREFHTMX
comment             : near-surface (usually, 2 meter) air temperature; derived f
> rom downscaled tasmax \& tasmin
units               : K
long_name           : Daily Near-Surface Air Temperature
standard_name       : air_temperature
_FillValue          : 1.0E20
_ChunkSizes         : 1
{\smallskip}
=== Metadata ===
Units          : K (original: K)
{\smallskip}
Shape          : [365, 600, 1440]
Data Type      : float
{\smallskip}
=== Variable Structure ===
Name: tas                       Type: float          
{\smallskip}
. 
. ///time variable
> ncdisp time using tas_day_BCC-CSM2-MR_ssp245_r1i1p1f1_gn_2050.nc
{\smallskip}
=== Dimensions ===
Dimension       Length   Coordinate     
time            365      [Yes]          
{\smallskip}
=== Scale/Offset Parameters ===
{\smallskip}
=== Attributes ===
units               : days since 2040-01-01
standard_name       : time
long_name           : time
calendar            : 365_day
axis                : T
_ChunkSizes         : 365
_CoordinateAxisType : Time
{\smallskip}
=== Metadata ===
Units          : days since 2040-01-01 (original: days since 2040-01-01)
{\smallskip}
Shape          : [365]
Data Type      : double
{\smallskip}
=== Variable Structure ===
Name: time                      Type: double         
{\smallskip}
. 
. ************6.2 Import Raster Data into Stata************
. 
. //Read the NetCDF file
. 
. ncread lon using tas_day_BCC-CSM2-MR_ssp245_r1i1p1f1_gn_2050.nc, clear 
Sucessfully import 1440 Obs into Stata.
{\smallskip}
. gen n=_n 
{\smallskip}
. qui sum n if lon>=108 \& lon<=115
{\smallskip}
. local lon_start = r(min)
{\smallskip}
. local lon_count = r(N)
{\smallskip}
. 
. ncread lat using tas_day_BCC-CSM2-MR_ssp245_r1i1p1f1_gn_2050.nc, clear 
Sucessfully import 600 Obs into Stata.
{\smallskip}
. gen n=_n 
{\smallskip}
. qui sum n if lat>=24 \& lat<=31
{\smallskip}
. local lat_start = r(min)
{\smallskip}
. local lat_count = r(N)
{\smallskip}
. 
. ncread tas using tas_day_BCC-CSM2-MR_ssp245_r1i1p1f1_gn_2050.nc, ///
>                         clear origin(1 `lat_start' `lon_start') ///
>                         size(-1 `lat_count' `lon_count')
{\smallskip}
{\smallskip}
Sucessfully import 286160 Obs into Stata.
{\smallskip}
. 
. gen date = time - 3650.5  + date("2050-01-01", "YMD")
{\smallskip}
. format date \%td
{\smallskip}
. 
. list in 1/10
{\smallskip}
     {\TLC}\HLI{51}{\TRC}
     {\VBAR}   time      lat       lon         tas        date {\VBAR}
     {\LFTT}\HLI{51}{\RGTT}
  1. {\VBAR} 3650.5   24.125   108.125   288.02673   01jan2050 {\VBAR}
  2. {\VBAR} 3650.5   24.125   108.375   289.05478   01jan2050 {\VBAR}
  3. {\VBAR} 3650.5   24.125   108.625   288.97476   01jan2050 {\VBAR}
  4. {\VBAR} 3650.5   24.125   108.875   288.53751   01jan2050 {\VBAR}
  5. {\VBAR} 3650.5   24.125   109.125   289.27686   01jan2050 {\VBAR}
     {\LFTT}\HLI{51}{\RGTT}
  6. {\VBAR} 3650.5   24.125   109.375   290.04214   01jan2050 {\VBAR}
  7. {\VBAR} 3650.5   24.125   109.625   289.90146   01jan2050 {\VBAR}
  8. {\VBAR} 3650.5   24.125   109.875   290.08304   01jan2050 {\VBAR}
  9. {\VBAR} 3650.5   24.125   110.125   286.86743   01jan2050 {\VBAR}
 10. {\VBAR} 3650.5   24.125   110.375   287.87402   01jan2050 {\VBAR}
     {\BLC}\HLI{51}{\BRC}
{\smallskip}
. 
. save "grid_all.dta", replace
file{\bftt{ grid_all.dta}} saved
{\smallskip}
. 
. ********6.3 Calculating Average temperature for hunan********
. 
. zonalstats tas_day_BCC-CSM2-MR_ssp245_r1i1p1f1_gn_2050.nc using "hunan.shp", /
> //
>                       var(tas) stats(avg) origin(1 `lat_start' `lon_start') //
> /
>                               size(1 `lat_count' `lon_count') crs(EPSG:4326) c
> lear
NetCDF variable 'tas' type: float
NetCDF variable 'tas' missing value attribute: 1.0E20 (type: float)
NetCDF CRS not detected. Using user-provided CRS: EPSG:4326
Coordinate systems are compatible, no reprojection needed
Created numeric variable: avg
Data successfully exported to Stata dataset.
{\smallskip}
. replace avg = avg - 273.15
(14 real changes made)
{\smallskip}
. save "hunan_temp.dta", replace
file{\bftt{ hunan_temp.dta}} saved
{\smallskip}
. 
. use hunan.dta, clear
{\smallskip}
. 
. rename Name z_Name  
{\smallskip}
. merge 1:1 z_Name using hunan_temp.dta,nogen
(variable {\bftt{z_Name}} was {\bftt{str11}}, now {\bftt{str16}} to accommodate using data's values)
{\smallskip}
    Result                      Number of obs
    \HLI{41}
    Not matched                             0
    Matched                                14  
    \HLI{41}
{\smallskip}
. save hunan_temp.dta, replace
file{\bftt{ hunan_temp.dta}} saved
{\smallskip}
. 
. geoframe create region ///
>    "hunan_temp.dta", id(_ID) centroids(_CX _CY) ///
>     shp(hunan_shp.dta) replace
(creating frame {\bftt{region}} from hunan_temp.dta)
(creating frame {\bftt{region_shp}} from hunan_shp.dta)
{\smallskip}
            Frame name: {\bftt{region}} [make current]
            Frame type: {\bftt{attribute}}
          Feature type: <none>
         Number of obs: {\bftt{14}}
               Unit ID: {\bftt{_ID}}
           Coordinates: {\bftt{_CX _CY}}
    Linked shape frame: {\bftt{region_shp}}
{\smallskip}
. 
. geoplot ///
>  (area region avg, color(Greys) ///
>                    level(6, quantile weight(avg))) ///
>  (line region, lwidth(vthin)), legend(position(sw))
{\smallskip}
. 
. graph save gragh4, replace
file {\bftt{gragh4.gph}} saved
{\smallskip}
. 
. ********6.4 Calculate 80km-radius IDW average temperatures for cities********
. use "grid_all.dta", clear
{\smallskip}
. rename lon ulon
{\smallskip}
. rename lat ulat
{\smallskip}
. gen n=_n
{\smallskip}
. save "grid_all_2.dta", replace
file{\bftt{ grid_all_2.dta}} saved
{\smallskip}
. 
. use "hunan_city.dta", clear
{\smallskip}
. matchgeop ORIG_FID lat lon using grid_all_2.dta, ///
>                      neighbors(n ulat ulon) within(80) gen(distance)
{\smallskip}
. 
. merge m:1 n using grid_all_2.dta, keep(3)
{\smallskip}
    Result                      Number of obs
    \HLI{41}
    Not matched                             0
    Matched                           148,555  (_merge==3)
    \HLI{41}
{\smallskip}
. drop _merge
{\smallskip}
. 
. list city ulat ulon distance date tas in 1/10
{\smallskip}
     {\TLC}\HLI{65}{\TRC}
     {\VBAR}     city     ulat      ulon    distance        date         tas {\VBAR}
     {\LFTT}\HLI{65}{\RGTT}
  1. {\VBAR} Yongzhou   25.125   111.625   72.995079   01jan2050   287.24945 {\VBAR}
  2. {\VBAR} Yongzhou   25.125   111.875   73.701637   01jan2050    284.2619 {\VBAR}
  3. {\VBAR} Chenzhou   25.125   113.125    76.69973   01jan2050   285.54437 {\VBAR}
  4. {\VBAR} Yongzhou   25.375   111.125   75.168007   01jan2050   285.06604 {\VBAR}
  5. {\VBAR} Yongzhou   25.375   111.375    56.90913   01jan2050   286.44476 {\VBAR}
     {\LFTT}\HLI{65}{\RGTT}
  6. {\VBAR} Yongzhou   25.375   111.625   45.657139   01jan2050   287.84082 {\VBAR}
  7. {\VBAR} Yongzhou   25.375   111.875   46.776157   01jan2050   286.04608 {\VBAR}
  8. {\VBAR} Yongzhou   25.375   112.125   59.573086   01jan2050   285.83316 {\VBAR}
  9. {\VBAR} Yongzhou   25.375   112.375   78.532799   01jan2050   286.43384 {\VBAR}
 10. {\VBAR} Chenzhou   25.375   112.625   70.880791   01jan2050    285.7345 {\VBAR}
     {\BLC}\HLI{65}{\BRC}
{\smallskip}
. 
. save "hunan_80km.dta", replace
file{\bftt{ hunan_80km.dta}} saved
{\smallskip}
. 
. drop if tas==.
(0 observations deleted)
{\smallskip}
. gen weight  = 1/distance
{\smallskip}
. gen weighted_tas = tas * weight
{\smallskip}
. bysort city date : egen sum_weighted_tas = total(weighted_tas)
{\smallskip}
. bysort city date : egen total_weight = total(weight)
{\smallskip}
. gen idw_tas = sum_weighted_tas / total_weight
{\smallskip}
. gen temp_c = idw_tas - 273.15
{\smallskip}
. 
. duplicates drop city date , force
{\smallskip}
Duplicates in terms of city date
{\smallskip}
(143,445 observations deleted)
{\smallskip}
. 
. list city  distance date  temp_c in 1/10
{\smallskip}
     {\TLC}\HLI{44}{\TRC}
     {\VBAR}    city    distance        date     temp_c {\VBAR}
     {\LFTT}\HLI{44}{\RGTT}
  1. {\VBAR} Changde   38.741573   01jan2050    7.65365 {\VBAR}
  2. {\VBAR} Changde   39.140995   02jan2050   7.220788 {\VBAR}
  3. {\VBAR} Changde   61.902889   03jan2050    8.44021 {\VBAR}
  4. {\VBAR} Changde   35.447414   04jan2050   6.136468 {\VBAR}
  5. {\VBAR} Changde    75.76696   05jan2050   7.520319 {\VBAR}
     {\LFTT}\HLI{44}{\RGTT}
  6. {\VBAR} Changde   63.030872   06jan2050   7.391534 {\VBAR}
  7. {\VBAR} Changde   39.613548   07jan2050   6.704889 {\VBAR}
  8. {\VBAR} Changde   58.544605   08jan2050   3.407739 {\VBAR}
  9. {\VBAR} Changde   39.613548   09jan2050   2.457117 {\VBAR}
 10. {\VBAR} Changde   75.469879   10jan2050   2.555414 {\VBAR}
     {\BLC}\HLI{44}{\BRC}
{\smallskip}
. 
. save "hunan_IDW.dta", replace
file{\bftt{ hunan_IDW.dta}} saved
{\smallskip}
. 
. ///IDW interpolated temperature distributions in Hunan
> use "hunan_IDW.dta" ,clear
{\smallskip}
. rename city Name
{\smallskip}
. merge m:1 Name using hunan.dta
{\smallskip}
    Result                      Number of obs
    \HLI{41}
    Not matched                             0
    Matched                             5,110  (_merge==3)
    \HLI{41}
{\smallskip}
. 
. keep if date == date("01jan2050", "DMY")
(5,096 observations deleted)
{\smallskip}
. save hunan_IDW_202500101.dta, replace
file{\bftt{ hunan_IDW_202500101.dta}} saved
{\smallskip}
. 
. geoframe create region ///
>     "hunan_IDW_202500101.dta", id(_ID) centroids(_CX _CY) ///
>     shp(hunan_shp.dta) ///
>     replace
(creating frame {\bftt{region}} from hunan_IDW_202500101.dta)
(creating frame {\bftt{region_shp}} from hunan_shp.dta)
{\smallskip}
            Frame name: {\bftt{region}} [make current]
            Frame type: {\bftt{attribute}}
          Feature type: <none>
         Number of obs: {\bftt{14}}
               Unit ID: {\bftt{_ID}}
           Coordinates: {\bftt{_CX _CY}}
    Linked shape frame: {\bftt{region_shp}}
{\smallskip}
. 
. geoplot ///
>     (area region temp_c , color(Greys) ///
>                 level(6, quantile weight(temp_c))) ///
>     (line region, lwidth(vthin)), legend(position(sw))
{\smallskip}
. 
. graph save gragh5, replace
file {\bftt{gragh5.gph}} saved
